
\chapter{Ringraziamenti}
\selectlanguage{italian}

        
Mentre scrivevo questa tesi è scoppiata la pandemia da COVID-19.
Le conseguenze in Italia 
sono state estremamente gravi, ma questa crisi ha portato con sé degli inaspettati e favorevoli sviluppi.
Internet mai più che in questo momento si è rivelato una risorsa fondamentale in quanto in tutto il paese,
improvvisamente, si è imposto lo smart-working, qualcosa che fino a prima del virus sembrava impossibile.
Migliaia di persone che si spostavano inutilmente ogni giorno adesso lavorano da casa, rendendo ancora più evidente
la quantità di risorse e di tempo che avremmo risparmiato se queste soluzioni fossero state adottate prima e
anche l'inquinamento è stato in gran parte abbattuto.


Ringrazio il Professore Enrico Bini per la sua preziosissima guida e il continuo supporto in questo percorso.

Ringrazio tutta la mia famiglia per avermi sempre sostenuto in ogni passo della mia vita, in particolare mio zio Piero che con la sua pazienza
è sempre stato al mio fianco.

Ringrazio  Valeria per essermi sempre vicina e motivarmi  e Silvia per aver reso una quarantena qualcosa di molto più piacevole.

Ringrazio tutti gli amici, per quello che gli amici fanno.

Ringrazio inoltre Reply Storm per avermi dato la possibilità di imparare tante nuove cose e
introdotto al mondo del lavoro.
In particolare, un ringraziamento sentito va a Piera Limonet e Girolamo Piccinni, che mi hanno seguito
e guidato scrupolosamente. 
Un grazie va anche ad Antonio e a tutto l'IoT team, che quotidianamente mi hanno aiutato a risolvere ogni problema.

